\documentclass[convert={density=300,size=1000x500,outext=.png}]{standalone}
\usepackage{tikz}
\usetikzlibrary{arrows}
\usepackage{amsmath}
\usepackage{booktabs}

\begin{document}
\begin{tikzpicture}[scale=0.12]
    \draw[black,line width=2pt] (0,0) rectangle (100, 50);

    \node[draw, rectangle, line width=1pt, align=center] (state) at (15,25) {First-order \\ planning state};

    %\node[draw, rectangle, align=center] (myNode) at (50,40) {Description Logics with \\ planning extensions};
    \node[draw, rectangle, line width=1pt, align=center] (feature) at (53,25) {
        \begin{tabular}{c}
            Description Logics with \\ planning extensions \\
            \hline
            \rule{0pt}{30pt}
            \begin{tikzpicture}[scale=0.12]
                \node[align=center] (dummy) at (0,40) {};
                \node[draw, rectangle, rounded corners=2pt, align=center] (root) at (0,34) {$\boldsymbol{\sqcap}$};
                \node[draw, rectangle, rounded corners=2pt, align=center] (n1) at (-10,26) {$\text{predicate}_\text{goal}$};
                \node[draw, rectangle, rounded corners=2pt, align=center] (n2) at (10,26) {$\boldsymbol{\neg}$};
                \node[draw, rectangle, rounded corners=2pt, align=center] (n3) at (10,18) {predicate};

                \draw[-latex] (dummy) -- (root);
                \draw[-latex] (root) -- (n1);
                \draw[-latex] (root) -- (n2);
                \draw[-latex] (n2) -- (n3);
            \end{tikzpicture}
        \end{tabular}};
    \node[draw, rectangle, line width=1pt, align=center] (valuation) at (90,25) {valuation};

    \draw[-latex, line width=1pt] (state) -- (feature);
    \draw[-latex, line width=1pt] (feature) -- (valuation);
\end{tikzpicture}
\end{document}